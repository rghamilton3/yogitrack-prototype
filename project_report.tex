\documentclass[11pt,a4paper]{article}
\usepackage[utf8]{inputenc}
\usepackage[margin=1in]{geometry}
\usepackage{tikz}
\usepackage{tikz-uml}
\usepackage{url}
\usepackage{hyperref}

\title{YogiTrack Studio Management System\\Part 1 Project Report}
\author{Robert Hamilton}
\date{\today}

\begin{document}

\begin{titlepage}
\centering
\vspace*{2cm}

{\Huge\bfseries YogiTrack Studio Management System}\\[0.5cm]
{\Large Part 1 Project Report}\\[2cm]

{\large Robert Hamilton}\\[1cm]

{\large \today}\\[3cm]

\begin{tabular}{ll}
\textbf{GitHub Repository:} & \url{https://github.com/rghamilton3/yogitrack-prototype} \\[0.5cm]
\textbf{Live Application:} & \url{https://yogitrack-prototype-a1daf9b97aac.herokuapp.com/}
\end{tabular}

\vfill
\end{titlepage}

\section{Project Overview}

YogiTrack is a MERN stack application designed for yoga studio management. The system enables administrators to manage instructors, customers, and classes with automated conflict detection and notification systems.

\section{Implemented Use Cases}

\subsection{Use Case 1: Add Instructor}
Allows adding new yoga instructors with duplicate name validation and automatic confirmation messaging to management.

\subsection{Use Case 2: Add Class}
Enables class creation with intelligent schedule conflict detection, alternative time slot suggestions, and instructor notifications.

\subsection{Use Case 4: Add Customer}
Facilitates customer registration with senior status tracking and comprehensive profile management.

\section{Architecture \& Design Decisions}

\subsection{Technology Stack}
\begin{itemize}
    \item \textbf{Frontend:} React 19 with functional components and hooks
    \item \textbf{Backend:} Express.js 5 with RESTful API design
    \item \textbf{Database:} MongoDB with Mongoose ODM
    \item \textbf{Build Tool:} Webpack with Babel transpilation
    \item \textbf{Deployment:} Heroku with GitHub Actions CI/CD
\end{itemize}

\subsection{Key Design Patterns}
\begin{itemize}
    \item Reusable EntityForm component eliminating code duplication
    \item Centralized API service layer with error handling
    \item Embedded document schemas for complex schedule management
    \item Conflict detection algorithms with alternative suggestions
\end{itemize}

\section{System Architecture}

\begin{figure}[h]
\centering
\begin{tikzpicture}
\tikzumlset{fill class=white}

% Frontend Layer
\umlclass[x=0,y=6]{React Frontend}{
+ EntityForm\\
+ Layout\\
+ Router
}{
+ handleCRUD()\\
+ validateInput()\\
+ displayNotifications()
}

% API Layer
\umlclass[x=6,y=6]{Express API}{
+ instructorRoutes\\
+ customerRoutes\\
+ classRoutes
}{
+ validateData()\\
+ checkConflicts()\\
+ sendNotifications()
}

% Database Layer
\umlclass[x=0,y=2]{MongoDB}{
+ instructors\\
+ customers\\
+ classes
}{
+ create()\\
+ read()\\
+ update()\\
+ delete()
}

% External Services
\umlclass[x=6,y=2]{External Services}{
+ Email Service\\
+ SMS Service
}{
+ sendConfirmation()\\
+ notifyInstructor()
}

% Relationships
\umluniassoc{React Frontend}{Express API}
\umluniassoc{Express API}{MongoDB}
\umluniassoc{Express API}{External Services}

\end{tikzpicture}
\caption{High-Level System Architecture}
\end{figure}

\section{Database Schema}

\begin{figure}[h]
\centering
\begin{tikzpicture}
\tikzumlset{fill class=white}

% Instructor Entity
\umlclass[x=0,y=6]{Instructor}{
- instructorId: String\\
- firstname: String\\
- lastname: String\\
- email: String\\
- phone: String\\
- bio: String\\
- active: Boolean
}{}

% Customer Entity
\umlclass[x=6,y=6]{Customer}{
- customerId: String\\
- firstname: String\\
- lastname: String\\
- email: String\\
- phone: String\\
- address: String\\
- senior: Boolean\\
- active: Boolean
}{}

% Class Entity
\umlclass[x=3,y=2]{Class}{
- classId: String\\
- className: String\\
- instructorId: String\\
- classType: String\\
- description: String\\
- payRate: Number\\
- daytime: [Daytime]\\
- active: Boolean
}{}

% Daytime Embedded Document
\umlclass[x=3,y=-2]{Daytime}{
- day: String\\
- time: String\\
- duration: Number
}{}

% Relationships
\umlassoc{Class}{Instructor}
\umlcompo{Class}{Daytime}

\end{tikzpicture}
\caption{Database Entity Relationship Diagram}
\end{figure}

\section{Key Implementation Features}

\subsection{Schedule Conflict Detection}
Implemented intelligent algorithm that checks for time overlaps and provides alternative scheduling suggestions when conflicts are detected.

\subsection{Automated Notifications}
Integrated confirmation messaging system that notifies managers and instructors upon successful operations.

\subsection{Responsive UI Components}
Created reusable React components with consistent styling and error handling across all entity management interfaces.

\subsection{CI/CD Pipeline}
Established automated deployment workflow using GitHub Actions with Heroku integration for continuous delivery.

\section{Conclusion}

The YogiTrack system successfully implements core studio management functionality with robust conflict detection, automated notifications, and a scalable MERN architecture. The reusable component design and comprehensive API layer provide a solid foundation for future feature expansion.

\end{document}